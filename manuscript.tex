\documentclass[author-year, review, 11pt]{components/elsarticle} %review=doublespace preprint=single 5p=2 column
%%% Begin My package additions %%%%%%%%%%%%%%%%%%%
\usepackage[hyphens]{url}
\usepackage{lineno} % add 
  \linenumbers % turns line numbering on 
\bibliographystyle{elsarticle-harv}
\biboptions{sort&compress} % For natbib
\usepackage{graphicx}
\usepackage{booktabs} % book-quality tables
%% Redefines the elsarticle footer
\makeatletter
\def\ps@pprintTitle{%
 \let\@oddhead\@empty
 \let\@evenhead\@empty
 \def\@oddfoot{\it \hfill\today}%
 \let\@evenfoot\@oddfoot}
\makeatother

% A modified page layout
\textwidth 6.75in
\oddsidemargin -0.15in
\evensidemargin -0.15in
\textheight 9in
\topmargin -0.5in
%%%%%%%%%%%%%%%% end my additions to header

\usepackage[T1]{fontenc}
\usepackage{lmodern}
\usepackage{amssymb,amsmath}
\usepackage{ifxetex,ifluatex}
\usepackage{fixltx2e} % provides \textsubscript
% use upquote if available, for straight quotes in verbatim environments
\IfFileExists{upquote.sty}{\usepackage{upquote}}{}
\ifnum 0\ifxetex 1\fi\ifluatex 1\fi=0 % if pdftex
  \usepackage[utf8]{inputenc}
\else % if luatex or xelatex
  \usepackage{fontspec}
  \ifxetex
    \usepackage{xltxtra,xunicode}
  \fi
  \defaultfontfeatures{Mapping=tex-text,Scale=MatchLowercase}
  \newcommand{\euro}{€}
\fi
% use microtype if available
\IfFileExists{microtype.sty}{\usepackage{microtype}}{}
\ifxetex
  \usepackage[setpagesize=false, % page size defined by xetex
              unicode=false, % unicode breaks when used with xetex
              xetex]{hyperref}
\else
  \usepackage[unicode=true]{hyperref}
\fi
\hypersetup{breaklinks=true,
            bookmarks=true,
            pdfauthor={},
            pdftitle={rgbif: a package for working with species occurrence data in R},
            colorlinks=true,
            urlcolor=blue,
            linkcolor=magenta,
            pdfborder={0 0 0}}
\urlstyle{same}  % don't use monospace font for urls
\setlength{\parindent}{0pt}
\setlength{\parskip}{6pt plus 2pt minus 1pt}
\setlength{\emergencystretch}{3em}  % prevent overfull lines
\setcounter{secnumdepth}{0}
% Pandoc toggle for numbering sections (defaults to be off)
\setcounter{secnumdepth}{0}
% Pandoc header



\begin{document}
\begin{frontmatter}

  \title{rgbif: a package for working with species occurrence data in R}
    \author[cstar]{Scott Chamberlain\corref{c1}}
   \ead{scott(at)ropensci.org} 
   \cortext[c1]{Corresponding author}
      \address[cstar]{University of California, Berkeley, CA, USA}    
  
  \begin{abstract}
  \begin{enumerate}
  \def\labelenumi{\arabic{enumi}.}
  \item
    xxx
  \item
    xxx
  \item
    xxx
  \item
    xxxx
  \end{enumerate}
  \end{abstract}
  
 \end{frontmatter}


\section{Introduction}\label{introduction}

Users of the popular statistical and mathematical computing platform R
(R Core Team 2014) enjoy a wealth of readily installable comparative
phylogenetic methods and tools.

To make the capabilities of NeXML available to R users in an easy-to-use
form, and to lower the hurdles to adoption of the standard, we present
RNeXML, an R package that aims to provide easy programmatic access to
reading and writing NeXML documents, tailored for the kinds of use-cases
that will be common for users and developers of the wealth of
evolutionary analysis methods within the R ecosystem.

\section{The rgbif package}\label{the-rgbif-package}

The \texttt{rgbif} package \ldots{}

\section{Conclusions and future
directions}\label{conclusions-and-future-directions}

\texttt{rgbif} \ldots{}

\subsection{Acknowledgements}\label{acknowledgements}

This project was supported in part by the Alfred P Sloan Foundation
(Grant 2013-6-22).

\subsection{Data Accessibility}\label{data-accessibility}

All software, scripts and data used in this paper can be found in the
permanent data archive Zenodo under the digital object identifier (DOI).
This DOI corresponds to a snapshot of the GitHub repository at
\href{https://github.com/sckott/msrgbif}{github.com/sckott/msrgbif}.

\section*{References}\label{references}
\addcontentsline{toc}{section}{References}

R Core Team. (2014). \emph{R: A language and environment for statistical
computing}. R Foundation for Statistical Computing, Vienna, Austria.
Retrieved from \url{http://www.R-project.org/}

\end{document}


